\begin{abstract}
Observing network data as a stream of relational events provides an opportunity to understand differences in how individuals interact over time.
For continuous-time network data, several approaches have recently been proposed for modeling dyadic event rates, conditioned on the observed history of events and nodal or dyadic covariates.  In many cases, however, interaction propensities -- and even the underlying mechanisms of interaction -- vary systematically across subgroups whose identities are unobserved.
For static networks, such heterogeneity has been treated via methods such as stochastic blockmodeling, which operate by assuming latent groups of individuals with similar tendencies in their group-wise interactions.
Here, we combine these two approaches by positing a latent partition of the vertex set such that event dynamics within and between node sets evolve in potentially distinct ways.
We illustrate the use of our model family by application to several forms of dyadic interaction data, including email communication and Twitter direct messages.
Parameter estimates from the fitted models clearly reveal heterogeneity in the dynamics among groups of individuals.
We also find that the fitted models have better predictive accuracy than either baseline models or relational event models without latent structure.
Our approach illustrates the utility of latent structure methods based on detailed dynamics, which can succeed even in the absence of differences in marginal interaction rates across groups. 
{ \color{red}  TODO: This last sentence is a bit unclear for a new reader. TODO: A few of theses sentences might be too long. TODO: stream vs sequence in first sentence?}
\end{abstract}
