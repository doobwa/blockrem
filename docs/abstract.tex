\begin{abstract}
Observing network data as a stream of relational events provides an opportunity to understand differences in how nodes interact given the past.
For continuous-time network data, a number of approaches have recently been proposed for modeling the rate of dyadic events conditioned on the observed history of events and covariates about the nodes.
For static networks, methods such as stochastic blockmodels have traditionally been used for node-level heterogeneity by assuming latent groups of individuals have similar tendencies in their group-wise interactions.
We propose to combine these two approaches by modeling the event dynamics within and between clusters of nodes.
The method is illustrated using dyadic interaction data such as email communication and Twitter direct messages.
Parameter estimates from the model clearly reveal heterogeneity in the dynamics among groups of individuals.
The fitted models have better predictive accuracy than baselines and
relational event models without the latent structure and a standard
stochastic blockmodel approach.
%Our approach illustrates the efficacy of combining a detailed model for local dependencies and a latent variable model for meso-scale dependencies.
\end{abstract}
